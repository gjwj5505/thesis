% !TEX root = thesis-middle.tex

\documentclass[ko]{snu-cse-bsc-thesis}

% Add your packages here, e.g.,
% \usepackage{tikz}
\usepackage{siunitx}

% For lorem ipsum; remove these lines when writing your thesis
\usepackage{lipsum}
\usepackage{jiwonlipsum}

% hyperref *must* be the last package to be loaded!
\usepackage[pdfusetitle]{hyperref}

\addbibresource{bib.bib}


\title{정적 분석기를 골탕 먹이는 입력 프로그램 합성 기술}
\titlealt{Program Synthesis for Attacking Static Analyzers}
\author{정원준}
\advisor{이광근}
\date{2025년 10월 29일}
\approvaldate{2026년 2월}

\koreankeywords{정적 분석, 프로그램 합성, 요약 해석, 난독화}
\englishkeywords{Static Analysis, Program Synthesis, Abstract Interpretation, Obfuscation}


\begin{document}
\maketitle

\pagenumbering{roman}
\begin{abstract}
  요약
\end{abstract}

\tableofcontents
\listoftables
\listoffigures

\chapter{서론}\label{chap:introduction}
\pagenumbering{arabic}
본 템플릿의 구성은 다음과 같다.
\ref{chap:body}장 본론의 \ref{sec:picture}절에서 그림의 예시를 보여준다.
\ref{sec:table}절에서 표의 예시를 보여준다.
\ref{chap:conclusion}장에서는 본 템플릿을 요약한다.

\section{절 예시}\label{sec:section}
\jiwon[2-3]


\chapter{본론}\label{chap:body}
정보 엔트로피는 각 메시지에 포함된 정보의 기댓값으로 식~\eqref{eq:entropy}\와 같다~\cite{6773024}.
\begin{equation}\label{eq:entropy}
  H(X) = -\sum_{i=1}^n {\mathrm{P}(x_i) \log_b \mathrm{P}(x_i)}
\end{equation}

\jiwon[4-6]


\section{그림}\label{sec:picture}
그림 예시는 그림~\ref{fig:example}\와 같다. 그림~\ref{fig:snu}\은 서울대학교 로고이고 그림~\ref{fig:eng}\는 서울대학교 공과대학 로고이다.

\begin{figure}[htp]
  \centering
  \begin{subfigure}[b]{0.5\textwidth}
    \centering
    \includegraphics[width=0.5\textwidth]{logo1.pdf}
    \bicaption{서울대학교 로고}{The logo of Seoul National University}\label{fig:snu}
  \end{subfigure}%
  \begin{subfigure}[b]{0.5\textwidth}
    \centering
    \includegraphics[width=0.9\textwidth]{logo2.pdf}
    \bicaption{공과대학 로고}{The logo of College of Engineering}\label{fig:eng}
  \end{subfigure}
  \bicaption[그림 예시 (목차 항목)]{그림 예시.}{An example of a figure.}\label{fig:example}
\end{figure}

\jiwon[7-8]


\section{표}\label{sec:table}
표 예시는 표~\ref{tab:example}\과 같다.\footnote{\jiwon[12]}

\begin{table}[htp]
  \centering
  \bicaption[표 예시 (목차 항목)]{표 예시.}{An example of a table.}\label{tab:example}
  \begin{tblr}{cc}
    \toprule
    상수 & 값 \\\midrule
    $c$ & \SI{299792458}{\meter\per\second} \\
    $h$ & \SI{6.62607015e-34}{\joule\per\hertz} \\\bottomrule
  \end{tblr}
\end{table}

\jiwon[9-10]


\chapter{결론}\label{chap:conclusion}
\jiwon[11]

\printbibliography

\begin{abstract}[en]
  \lipsum[1]
\end{abstract}
\end{document}
